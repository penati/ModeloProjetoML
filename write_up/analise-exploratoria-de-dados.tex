\documentclass[]{article}
\usepackage{lmodern}
\usepackage{amssymb,amsmath}
\usepackage{ifxetex,ifluatex}
\usepackage{fixltx2e} % provides \textsubscript
\ifnum 0\ifxetex 1\fi\ifluatex 1\fi=0 % if pdftex
  \usepackage[T1]{fontenc}
  \usepackage[utf8]{inputenc}
\else % if luatex or xelatex
  \ifxetex
    \usepackage{mathspec}
  \else
    \usepackage{fontspec}
  \fi
  \defaultfontfeatures{Ligatures=TeX,Scale=MatchLowercase}
\fi
% use upquote if available, for straight quotes in verbatim environments
\IfFileExists{upquote.sty}{\usepackage{upquote}}{}
% use microtype if available
\IfFileExists{microtype.sty}{%
\usepackage{microtype}
\UseMicrotypeSet[protrusion]{basicmath} % disable protrusion for tt fonts
}{}
\usepackage[margin=1in]{geometry}
\usepackage{hyperref}
\hypersetup{unicode=true,
            pdftitle={Lesson 1 - Análise exploratória de dados},
            pdfauthor={Filipe Penati},
            pdfborder={0 0 0},
            breaklinks=true}
\urlstyle{same}  % don't use monospace font for urls
\usepackage{color}
\usepackage{fancyvrb}
\newcommand{\VerbBar}{|}
\newcommand{\VERB}{\Verb[commandchars=\\\{\}]}
\DefineVerbatimEnvironment{Highlighting}{Verbatim}{commandchars=\\\{\}}
% Add ',fontsize=\small' for more characters per line
\usepackage{framed}
\definecolor{shadecolor}{RGB}{248,248,248}
\newenvironment{Shaded}{\begin{snugshade}}{\end{snugshade}}
\newcommand{\KeywordTok}[1]{\textcolor[rgb]{0.13,0.29,0.53}{\textbf{#1}}}
\newcommand{\DataTypeTok}[1]{\textcolor[rgb]{0.13,0.29,0.53}{#1}}
\newcommand{\DecValTok}[1]{\textcolor[rgb]{0.00,0.00,0.81}{#1}}
\newcommand{\BaseNTok}[1]{\textcolor[rgb]{0.00,0.00,0.81}{#1}}
\newcommand{\FloatTok}[1]{\textcolor[rgb]{0.00,0.00,0.81}{#1}}
\newcommand{\ConstantTok}[1]{\textcolor[rgb]{0.00,0.00,0.00}{#1}}
\newcommand{\CharTok}[1]{\textcolor[rgb]{0.31,0.60,0.02}{#1}}
\newcommand{\SpecialCharTok}[1]{\textcolor[rgb]{0.00,0.00,0.00}{#1}}
\newcommand{\StringTok}[1]{\textcolor[rgb]{0.31,0.60,0.02}{#1}}
\newcommand{\VerbatimStringTok}[1]{\textcolor[rgb]{0.31,0.60,0.02}{#1}}
\newcommand{\SpecialStringTok}[1]{\textcolor[rgb]{0.31,0.60,0.02}{#1}}
\newcommand{\ImportTok}[1]{#1}
\newcommand{\CommentTok}[1]{\textcolor[rgb]{0.56,0.35,0.01}{\textit{#1}}}
\newcommand{\DocumentationTok}[1]{\textcolor[rgb]{0.56,0.35,0.01}{\textbf{\textit{#1}}}}
\newcommand{\AnnotationTok}[1]{\textcolor[rgb]{0.56,0.35,0.01}{\textbf{\textit{#1}}}}
\newcommand{\CommentVarTok}[1]{\textcolor[rgb]{0.56,0.35,0.01}{\textbf{\textit{#1}}}}
\newcommand{\OtherTok}[1]{\textcolor[rgb]{0.56,0.35,0.01}{#1}}
\newcommand{\FunctionTok}[1]{\textcolor[rgb]{0.00,0.00,0.00}{#1}}
\newcommand{\VariableTok}[1]{\textcolor[rgb]{0.00,0.00,0.00}{#1}}
\newcommand{\ControlFlowTok}[1]{\textcolor[rgb]{0.13,0.29,0.53}{\textbf{#1}}}
\newcommand{\OperatorTok}[1]{\textcolor[rgb]{0.81,0.36,0.00}{\textbf{#1}}}
\newcommand{\BuiltInTok}[1]{#1}
\newcommand{\ExtensionTok}[1]{#1}
\newcommand{\PreprocessorTok}[1]{\textcolor[rgb]{0.56,0.35,0.01}{\textit{#1}}}
\newcommand{\AttributeTok}[1]{\textcolor[rgb]{0.77,0.63,0.00}{#1}}
\newcommand{\RegionMarkerTok}[1]{#1}
\newcommand{\InformationTok}[1]{\textcolor[rgb]{0.56,0.35,0.01}{\textbf{\textit{#1}}}}
\newcommand{\WarningTok}[1]{\textcolor[rgb]{0.56,0.35,0.01}{\textbf{\textit{#1}}}}
\newcommand{\AlertTok}[1]{\textcolor[rgb]{0.94,0.16,0.16}{#1}}
\newcommand{\ErrorTok}[1]{\textcolor[rgb]{0.64,0.00,0.00}{\textbf{#1}}}
\newcommand{\NormalTok}[1]{#1}
\usepackage{graphicx,grffile}
\makeatletter
\def\maxwidth{\ifdim\Gin@nat@width>\linewidth\linewidth\else\Gin@nat@width\fi}
\def\maxheight{\ifdim\Gin@nat@height>\textheight\textheight\else\Gin@nat@height\fi}
\makeatother
% Scale images if necessary, so that they will not overflow the page
% margins by default, and it is still possible to overwrite the defaults
% using explicit options in \includegraphics[width, height, ...]{}
\setkeys{Gin}{width=\maxwidth,height=\maxheight,keepaspectratio}
\IfFileExists{parskip.sty}{%
\usepackage{parskip}
}{% else
\setlength{\parindent}{0pt}
\setlength{\parskip}{6pt plus 2pt minus 1pt}
}
\setlength{\emergencystretch}{3em}  % prevent overfull lines
\providecommand{\tightlist}{%
  \setlength{\itemsep}{0pt}\setlength{\parskip}{0pt}}
\setcounter{secnumdepth}{0}
% Redefines (sub)paragraphs to behave more like sections
\ifx\paragraph\undefined\else
\let\oldparagraph\paragraph
\renewcommand{\paragraph}[1]{\oldparagraph{#1}\mbox{}}
\fi
\ifx\subparagraph\undefined\else
\let\oldsubparagraph\subparagraph
\renewcommand{\subparagraph}[1]{\oldsubparagraph{#1}\mbox{}}
\fi

%%% Use protect on footnotes to avoid problems with footnotes in titles
\let\rmarkdownfootnote\footnote%
\def\footnote{\protect\rmarkdownfootnote}

%%% Change title format to be more compact
\usepackage{titling}

% Create subtitle command for use in maketitle
\newcommand{\subtitle}[1]{
  \posttitle{
    \begin{center}\large#1\end{center}
    }
}

\setlength{\droptitle}{-2em}

  \title{Lesson 1 - Análise exploratória de dados}
    \pretitle{\vspace{\droptitle}\centering\huge}
  \posttitle{\par}
    \author{Filipe Penati}
    \preauthor{\centering\large\emph}
  \postauthor{\par}
      \predate{\centering\large\emph}
  \postdate{\par}
    \date{24 de Junho, 2018}


\begin{document}
\maketitle

\begin{Shaded}
\begin{Highlighting}[]
\KeywordTok{load}\NormalTok{(}\StringTok{"analise-exploratoria-de-dados.RData"}\NormalTok{)}
\end{Highlighting}
\end{Shaded}

\section{Data}\label{data}

Aqui está uma olhada em nossos dois data frames. O primeiro apresenta o
registro de um experimento onde foi observado o peso de 50 pintinhos,
nos 21 primeiros dias de vida, submetidos a 4 dietas diferentes. O
segundo, é um subconjunto onde é apresentado o peso dos pintinhos ao
final dos 21 dias. Apenas 45 dos 50 pintinhos foram acompanhados até 21º
dia.

\begin{Shaded}
\begin{Highlighting}[]
\KeywordTok{head}\NormalTok{(data)}
\end{Highlighting}
\end{Shaded}

\begin{verbatim}
##   weight Time Chick Diet
## 1     42    0     1    1
## 2     51    2     1    1
## 3     59    4     1    1
## 4     64    6     1    1
## 5     76    8     1    1
## 6     93   10     1    1
\end{verbatim}

\begin{Shaded}
\begin{Highlighting}[]
\KeywordTok{head}\NormalTok{(time_}\DecValTok{21}\NormalTok{)}
\end{Highlighting}
\end{Shaded}

\begin{verbatim}
##   weight Time Chick Diet
## 1    205   21     1    1
## 2    215   21     2    1
## 3    202   21     3    1
## 4    157   21     4    1
## 5    223   21     5    1
## 6    157   21     6    1
\end{verbatim}

\section{Figures}\label{figures}

Essa figura apresenta os registros do peso dos pintinhos de acordo com a
dieta.

\begin{Shaded}
\begin{Highlighting}[]
\NormalTok{data.plot}
\end{Highlighting}
\end{Shaded}

\begin{center}\includegraphics{analise-exploratoria-de-dados_files/figure-latex/unnamed-chunk-3-1} \end{center}

Aqui é apresentado a distribuição do peso dos pintinhos ao final dos 21
dias, de acordo com a dieta.

\begin{Shaded}
\begin{Highlighting}[]
\NormalTok{time_}\FloatTok{21.}\NormalTok{plot}
\end{Highlighting}
\end{Shaded}

\begin{center}\includegraphics{analise-exploratoria-de-dados_files/figure-latex/unnamed-chunk-4-1} \end{center}

\section{Descriptive Statistics}\label{descriptive-statistics}

Quando resumo da evolução do peso agrupando de acordo com os dias de
vida, observamos como a variância aumenta com os dias. Podemos ver, com
isso, a grande influência das dietas sobre o peso dos pintinhos.

\begin{Shaded}
\begin{Highlighting}[]
\NormalTok{weight_by_time}
\end{Highlighting}
\end{Shaded}

\begin{verbatim}
## # A tibble: 12 x 5
##     Time weight_mean weight_sd weight_min weight_max
##    <dbl>       <dbl>     <dbl>      <dbl>      <dbl>
##  1     0        41.1      1.13         39         43
##  2     2        49.2      3.69         35         55
##  3     4        60.0      4.50         48         69
##  4     6        74.3      9.01         51         96
##  5     8        91.2     16.2          57        131
##  6    10       108.      24.0          51        163
##  7    12       129.      34.1          54        217
##  8    14       144.      38.3          68        240
##  9    16       168.      46.9          71        287
## 10    18       190.      57.4          72        332
## 11    20       210.      66.5          76        361
## 12    21       219.      71.5          74        373
\end{verbatim}

Quando olhamos o resumo do peso final dos pintinhos pelas 4 dietas, fica
mais claro a diferença entre os resultados médios de cada dieta.
Entretanto, ainda podemos ver um grande variabilidade nos resultados das
dietas. A que se mostra mais consistente é a dieta ``4'', apresentando
um desvio médio de 43.3 gramas.

\begin{Shaded}
\begin{Highlighting}[]
\NormalTok{weight_21days_by_diet}
\end{Highlighting}
\end{Shaded}

\begin{verbatim}
## # A tibble: 4 x 5
##   Diet  weight_mean weight_sd weight_min weight_max
##   <fct>       <dbl>     <dbl>      <dbl>      <dbl>
## 1 1            178.      58.7         96        305
## 2 2            215.      78.1         74        331
## 3 3            270.      71.6        147        373
## 4 4            239.      43.3        196        322
\end{verbatim}

Por fim, para evidenciar a diferença entre o comportamento das dietas ao
longo dos 21 dias, foi desenhado o gráfico abaixo de acordo com as
médias para cada dia e dieta.

\begin{Shaded}
\begin{Highlighting}[]
\NormalTok{weight_time_diet.plot}
\end{Highlighting}
\end{Shaded}

\begin{center}\includegraphics{analise-exploratoria-de-dados_files/figure-latex/unnamed-chunk-7-1} \end{center}


\end{document}
